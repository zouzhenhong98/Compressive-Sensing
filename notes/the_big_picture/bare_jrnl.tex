\documentclass{article}[11pt]
\usepackage{ctex}
\usepackage[colorlinks,linkcolor=blue]{hyperref}
\title{Compressive Sensing: The Big Picture\\--学习笔记}
\author{邹镇洪
\thanks{联系:北京市海淀区北京航空航天大学数学科学学院,100084}
\thanks{邮箱:zouzhenhong@buaa.edu.cn}}
\date{\today}
\usepackage[a4paper,left=15mm,right=15mm,top=15mm,bottom=15mm]{geometry}  
%https://blog.csdn.net/meiqi0538/article/details/82887300

\begin{document}
\maketitle

\begin{abstract}
本文是对\href{https://sites.google.com/site/igorcarron2/cs}{Compressive Sensing: The Big Picture}的学习笔记。事实上,这是一个资源网站,集合了2013年以前的CS学习资源,包括压缩感知每个部分的相关论文和网站,建议作为一个检索、补充学习的学习网站。\par
\href{https://sites.google.com/site/igorcarron2/compressedsensingclasses}{The online CS course list in 2010-2012}
\end{abstract}


\section{Overview}
\paragraph{Overview} Compressed Sensing(CS) or Compressive Sensing is about \textbf{acquiring and recovering a sparse signal} in the most efficient way possible (subsampling) with the help of an \textbf{incoherent projecting basis}. Compressed Sensing provides a new framework for acquiring sparse signals in a mutiplexed manner. The main theoretical findings in this recent field have mostly centered on how many multiplexed measurements are necessary to reconstruct the original signal and the attendant nonlinear reconstruction techniques needed to demultiplex these signals.
\paragraph{Two emerging problems} the ability to search for \textbf{bases or dictionaries} in which sets of signals can be decomposed in a sparse manner and the ability to find and quantify specific measurements tools that are incoherent with said dictionaries.(find the basis and reconstruct the sparse signal) There are theoretical results yielding the minimum number of required measurements needed to produce the original signal given a specific pair of measurement matrices and nonlinear solvers. In all cases, the expected number of compressed measurements is expected to be low relative to traditional Nyquist sampling constraints.


\section{To Begin}
If you are new to CS, refer to this section "1. Understanding Compressed Sensing" where there are various links for a quick or deeper understanding of the concept.

\section{Dictionaries for Sparse Recovery (How to Make or Find them}
\section{Compressed Sensing Measurement / Sparse Signal Encoding}
\section{•}

\end{document}